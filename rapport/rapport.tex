% Une ligne commentaire débute par le caractère « % »

\documentclass[a4paper]{article}

% Options possibles : 10pt, 11pt, 12pt (taille de la fonte)
%                     oneside, twoside (recto simple, recto-verso)
%                     draft, final (stade de développement)

\usepackage[utf8]{inputenc}   % LaTeX, comprends les accents !
\usepackage[T1]{fontenc}      % Police contenant les caractères français
\usepackage[francais]{babel}
\usepackage{fullpage}
\usepackage{multicol}
\usepackage{hyperref}
\hypersetup{
    colorlinks=true,
    linkcolor=blue,
    filecolor=magenta,      
    urlcolor=red
    }
% \urlstyle{same} % ça sert à rien ce truc
\usepackage{bookmark}
\usepackage{blindtext}



\usepackage{graphicx}  % pour inclure des images
\graphicspath{ {rapport/img/} }

%\pagestyle{headings}        % Pour mettre des entêtes avec les titres
                              % des sections en haut de page

 \title{  TP2 : Les capteurs\\         % Les paramètres du titre : titre, auteur, date
  Programmation mobile}
\author{Mohamad Satea Almallouhi - Tony Nguyen\\
  \emph{M1 Génie Logiciel}\\
  Faculté des Sciences\\
Université de Montpellier.}
\date{5 mars 2024}



\begin{document}
    \maketitle
    \begin{center}
    \includegraphics[width=\textwidth]{makima}
    \end{center}

    % \begin{abstract}     % Résumé du travail
    %   \emph{Description très succinte du problème et des différentes étapes de réalisation}
    % \end{abstract}
    \newpage
    %\dominitoc  % initializer les minitoc
    \tableofcontents

    \newpage
    \begin{multicols}{2}
        [
            Faire une vidéo, rapport+read.md(instruction) screenchot résultats + code. +bonus bien fait,beau,tests,Kotlin,latex
        ]
        \section*{Introduction}
        \addcontentsline{toc}{section}{Introduction}
        \paragraph{}
            Dans ce TP, nous allons l'utilisation des capteurs intégré dans nos smartphones.
            
            Nous allons voir comment manipuler les différents types de capteur comme le GPS, la boussole, le gyroscope, etc ...
        \paragraph{}
            Les sections du rapport suit les exercices.
        \section*{Démonstration}
        \addcontentsline{toc}{section}{Démonstration}
            \paragraph{}
            En ligne sur Youtube, à l'adresse URL \url{https://youtu.be/nQUkpSUjJlY} une démonstration vidéo de notre travail.
        \section{Liste des capteurs}
        \section{Détection de la disponibilité des capteurs}  
        \section{Accéléromètre}
        \section{Direction}
        \section{Secouer un appareil}
        \section{Proximité}
        \section{Géolocalisation}
    \end{multicols}
\end{document}